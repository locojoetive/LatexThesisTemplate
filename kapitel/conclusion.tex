\chapter{Conclusion}
Paper Fighters enables multi-level modelling of two-dimensional platformer, with a set of predefined game objects. It provides the functionality of creating a stage within the LML-Editor and simulate the result.
It enables creating a complete level with several stages.
However, the plug-in does not support positioning the game objects and defining the size of the level graphically inside the model. Instead, it is only possible to set them directly in the properties view, what decreases its usability and ease of use. Moreover, the painting application contained in our plug-in is not able to assign the instantiated objects to the model, and create the simulation modelling is not possible, due to the missing conversion of the strings received from the client to the LML model entities.
Nevertheless, the result of our research is clearly the recommendation of using model-driven software development for creating complex software artefacts and systems. This extends to video games, when considering, that their complexity is dramatically increasing with upcoming technologies to provide innovative and never experienced game components to the player \cite{Blow:2004:GDH:971564.971590}. Since prototyping and testing are identified as the key stones in an iterative game development process\cite{reyno2009automatic}, the automization of creating prototypes through MDD, with real-time communication between a model and its simulation is proposed, as it enhances quality and validity of both. Especially, if a game is supposed to have a sequel or if other games of the same genre are considered in production, using MDGD is preferred. Due to the reusability of these objects the development process of future project is enhanced. All video games with respect to some exceptions, can be modeled with each game component instance considered to be an entity with specific attributes and specific relations to other entities. Hence, the model-driven approach with a DSL for developing model simulation in combination with simulation modelling for games is still recommended, due to the numerous advantages, like accelerating speed and increasing the productivity of the development effectively.