
\chapter{Introduction}
\section{Motivation}
Complexity in video games is dramatically increasing to provide innovative and never experienced game components to the player \cite{Blow:2004:GDH:971564.971590}. Since prototyping and testing are identified as the key stones in an iterative game development process, the automization of creating prototypes through model-driven development is proposed\cite{reyno2009automatic}. Model-driven software development has become a fundamental technique in software engineering. As it is beneficial to be able to simulate a system's model---for validation, addition, or removal of certain requirements---it is desirable to add the functionality of model simulation to a certain model. Domain-specific languages, which are based on the principles of deep modelling, are used more widely in combination with model-driven software design, due to the ability of simplefying specific parts of a system. They are preffer The release of a considerable number of sophisticated simulation software, which enables precise simulation of dynamic systems and code generation, and the recent generation of real time simulators, led model driven design approaches to become relevant in the recent past. This study discusses, whether a DSL created in Melanee can be extended to support model simulation for an external (not Java based) game engine. Therefore, we create a meta-model in Melanee, which represents our modelling language, develop a simulation in the form of a Unity application, and a Melanee plug-in, which contributes as the simulation's interface. The simulation is planed to be in form of a two dimensional platformer running on the Unity game engine.
\section{Organisation}
This study is structured as follows.
The next section (Foundations) introduces relevant foundations for this study, including fundamentals of deep modelling and model-driven software development. We will examine the current state of model-driven approaches and disscuss their role in software development nowadays. Furthermore we will focus on Melanee, a workbench for creating domain specific languages, and the frameworks forming its basis. Those are the Eclipse Modelling Framework, the Graphical Modeling Framework and the Eclipse Project, which  Melanee is built on. Last in this section we will outline the Unity game engine. Subsequently we describe \textit{Paper Figthers} (Contribution), a 2D Platformer based on a multi-level model.  We will present the components it is built on. Afterwards we introduce related works, and how the tool we develop differs from them (Related Works). Then limitations and potential extensions (Limitations and Future Work) of our tool are discussed. Finally we will disscuss the results of this work (Conclusion).
