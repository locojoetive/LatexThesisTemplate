\chapter{Related Work}
Prototyping and testing became the key motivation for Model-driven development with domain specific languages. Thus, it is becoming increasingly relevant in game design and development. Whereas quite a few games allow the player to instantiate predefined object from a game model to share it with the game's community, it is surprising that real-time modelling in combination with multi-level modelling is rarely reflected upon \cite{Noah}. In this section we will show an example for a multi-level modeling based game, a number of MDGD approaches and two games sharing the benefits of model-driven level design with their players.
\section[Eine Deep-Model-Basierte Echtzeitsimulation]{Die Implementierung von GeoWars\cite{Noah}}
This thesis deals with the implementation of Geowars, a two-dimensional multi-level fighting game simulation for three players, developed by Noah Metzger from the Chair of Software Engineering at the University of Mannheim. By Using the Melanee LML editor a player can model a game level and share it through the internet. Afterwards two players can play the level against each other on one computer. The simulation feature allows modeling the current state of the played level, its objects, its players and, since the Triggerhappy-DLC also the players projectiles. Hence, it enables the simulation of a model, and real-time modification, which makes the outcome of a game more unpredictable. These principles applied to game development in general will enhance the validity of testing and prototyping.

\section{Automatic Prototyping In Model-Driven Game Development\cite{reyno2009automatic}}
This paper discusses how MDGD enhances the development process of video games by increasing the productivity and accelerating the design process. Reyno et. al. hold the belief that By abstracting the code that implements games, games are specified even more precise. To observe their statements they develop a prototype of a 2D platformer by manually programming it and compare the needed time to generating 94\% of the required code from a model with the MOFscript transformation. Instead of spending 7 days while programming it manually, they only required a few hours for generation from the model and moreover, were able to develop a second prototype by reusing the complete model from the first prototype.

\section{Super Mario Maker\cite{NIN}}
By developing a tool consisting of almost all components ever used in a two-dimensional Super Mario game and giving the opportunity to combine them with each other to the player, Nintendo produced a game that enables level modelling to its consumers. One could assume that by being the most successful platformer series ever produced the success of this project was not surprising. However, the ease of use of the level editor is also responsible for its success. The simple drag and drop function makes it easy to learn how to build stages, and if that doesn't suit the player he can simply play countless levels created from people around the world. The original use of this tool was to create Super Mario Games. This shows how the reusability of model components due to MDGD can increase the return of investment remarkably.

\section{Ultimate Chicken Horse\cite{CEG}}
Ultimate Chicken Horse is a two player 2D platformer game developed by Clever Endeavour, an indie game studio from Montreal, Canada. It combines simple level design functionality with competitive gameplay. "It's a fine balance between being an awesome level designer, and being a huge jerk"\cite{CEG}
As well as Super Mario Maker, it shows that not just the development but also the accessibility of model components to the user can be an appealing offer.